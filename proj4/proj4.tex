\documentclass[a4paper,11pt,draft]{article}
\usepackage[czech]{babel}
\usepackage[left=2cm,top=3cm,
textheight=24cm,textwidth=17cm]{geometry}
\usepackage[T1]{fontenc}
\usepackage[utf8]{inputenc}
\usepackage{times}
\usepackage{amsmath}
\usepackage{amsthm}
\usepackage{amsfonts}
\usepackage[hyphens]{url}
\usepackage{hyperref}



\urlstyle{rm}
\begin{document}
	\begin{titlepage}
		\setcounter{page}{0}
		\begin{center}	
			\addtolength{\parskip}{0.6em}
			{\Huge
				\textsc{
					Vysoké učení technické v Brně}}
			
			{\huge 
				\textsc{
					Fakulta informačních technologií}}
			
			\vspace{\stretch{0.382}}
			{\LARGE
				Typografie a publikování\,--\,4. projekt
				
				\Huge Závěrečná fáze vědeckého výzkumu }
			
			\vspace{\stretch{0.618}}
		\end{center}
		{\Large 
			\today \hfill Danil Domrachev}
	\end{titlepage}

	\newpage
	\raggedright
	
	\section{Vydání článků a monografií}
	
	Samotný závěr naučného výzkumu ještě není závěrém celé práce vědce. Aby ukázal, že něco dosahl, případně mohl sdělit svou práci dalším vyzkumníkům, musí ji pořádně upravit a vytisknout (Buď fyzicky nebo elektronicky).
	
	\begin{quotation}
	\centering
	Vydání monografie - pravidelná závěrečná fáze prakticky
	jakéhokoli úspěšného významného vědeckého výzkumu\footnote{Volně přeložená citace - viz \cite{clserpub1}}
	\end{quotation}

	\section{Kratká historie}
	
	\subsection{Jak to bylo minule}
	Šíření informace v minulých stoletích nebylo tak rychlé a jednoduché. Zde \cite{el2} o tom můžete přečíst více. V~době kdy se jen zarodil první vědecký časopis bylo potřeba pracovat s odborníky z oblasti typografie, což vyžadovalo určitý čas a peníze.
	

	\subsection{Co máme teď}
	
	Postupem času činnost vydání se stává poměrně jednoduchou. Proces publikace postupně přechází do elektronické podoby. Je možné v jediný okamžik najít spoustu příkladů úspěšně obhajených vědeckých textů buď na téma z fizyky\,\cite{TRAVNICEK2021thesis}, matematiky\,\cite{Trinh_Viet2020thesis}, nebo jakéhokoliv dalšího. 
	
	Druhou příčinou usnadnění vydání článků a monografií se stal rozvoj výpočetní techniky jelikož, jak se můžeme dozvědět z \cite{serpub}, typografie vždy těsně souvisela s výpočty.

	\subsection{Pohled do budoucna}
	
	Úlohou nadcházejících let je optimalizace existujících procesů a technologií použitých při vydání publikací, rozvoj nových standardů\footnote{O hlavnách aspektech v oblasti typografie - viz\;\cite{el3}}. V~této monografii \cite{mon2}, se dozvíte, že pomoct v tom můžou technologie automatického rozpoznávání textů nebo banální zlepšení designu klávesnice pro pohodlný přístup k specifičtějším symbolům.


	\section{Kvalita vědecké práce}
	
	Můžeme předpokládat, že s šířením tiskáren a dále i s objevem takových elektronických nástrojů jako \LaTeX, lidé přirozeně dávají méně pozoru na to co sázejí ve prospěch rychlosti. Tím ale vzníkají hloupé chyby a překlepy které kází dojem čtenáře. Na to jak bude text chápan taky má vliv styl podávání odborného materiálu, podrobněji viz. \cite{mon1}.
	
	Důležitou roli při ocenění kvality také hraje samotná osoba (to je jeden z faktorů o kterých se můžeme dozvědět~z~\cite{clserpub2}). Počet výzkumných prácí se zvětšuje, kompetence jejich autorů ale ve velké části případů se~dá zpochybnit.
	
	
	\section{Problémy publikace}
	
	Řekneme, že publikace získala vysoké ocenění a publikuje se v nějakém uznáváném časopisu. Bohužel bude dostupná ne tak velkému počtu lidí jak by se chtělo. Problém spočívá v tom, že předplatné za většinu víceméně autoritativních časopisu může si povolit ne každý běžný člověk. Tento fakt potvrzuje například~\cite{el1}. Odtud se můžete dozděvět i o dalších problémech při publikaci.

	
	
	
	
	\newpage
	\bibliographystyle{plain}
	\bibliography{ref}

\end{document}